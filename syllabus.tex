\documentclass{article}
\usepackage{graphicx}
\usepackage[margin=1in]{geometry}
\usepackage{fancyhdr}
\pagestyle{fancy}
\usepackage{xcolor}
\usepackage[hidelinks]{hyperref}

\begin{document}

\section*{PL PATH 875: Phylogenetic Practicum}

Spring 2026: Friday 10:00-10:50a

\subsubsection*{Instructor}
\begin{itemize}
    \item Claudia Solis-Lemus, PhD
    \item email: solislemus@wisc.edu
    \item website: \url{https://solislemuslab.github.io/}
    \item office hours: By appointment
\end{itemize}


\subsubsection*{Course description}

An advanced course in the practice of phylogenetic inference from DNA sequence data by fully reproducing the data analysis of a published paper.

\begin{itemize}
    \item Credits: 3
    \item Level: Advanced
    \item Breadth: Biological Science
    \item L\&S Credit Type: Counts as LAS credit (L\&S)
    \item Course Options: 50\% Graduate Coursework Requirement
    \item This class meets for one 50-minute class period each week over the Spring semester and carries the expectation that students will work on course learning activities (reading, writing, problem sets, studying, etc) for about 3 hours out of the classroom for every class period. The syllabus includes more information about meeting times and expectations for student work
    \item Modality: In-person
    \item Grading scale: A-F
    \item Class website: \url{https://github.com/solislemuslab/phylo-practicum}
\end{itemize}


\subsubsection*{Requisites}
BIOLOGY/BOTANY/ZOOLOGY 151, BIOLOGY/BOTANY 130, BIOLOGY/ZOOLOGY 101, BIOCORE 381, STATS 240, 301, or 371, or graduate/professional standing


\subsubsection*{Learning outcomes}

By the end of the course, the student will be able to

\begin{enumerate}
    \item Reproduce all the steps of a phylogenomics data analysis of a published paper
    \item Learn best computing and reproducibility practices for phylogenomic data analyses
\end{enumerate}


\subsubsection*{Main topics}
\begin{itemize}
    \item Introduction to reproducible scripting and programming
    \item Sequence alignment, filtering, orthology
    \item Basics of gene tree building
    \item Basics of species tree building
    \item Molecular dating and divergence time estimation
    \item Handling hybridization and/or multiple gene copies
    \item Pitfalls in phylogenomics: practical guidelines when analyzing big data
\end{itemize}


\subsubsection*{Regular and Substantive Interaction}
\begin{itemize}
    \item Participation in regularly scheduled learning sessions (where there is an opportunity for direct interaction between the student and the qualified instructor).
    \item Provide personalized comments for an individual student's assignments and data analyses.
    \item Actively facilitate in-class discussion of relevant papers in phylogenomics.
    \item Instructor posts announcements, email, or slack check-ins about academic aspects of the class.
    \item Identify students struggling to reach mastery through observation of discussion activity, assessment completion, or even user activity and offer additional opportunities for interaction.
    \item Use of small working/study groups that are moderated by the instructor.
\end{itemize}


\subsubsection*{Grading}

Final reproducible script (100 points). Reproducible script grading rubric:
\begin{itemize}
    \item \textbf{inadequate (50-69):} Just a copy-paste of commands without any explanations
    \item \textbf{adequate (70-79):} Sequence of commands interleaved with comments, but important details are missing that makes it impossible for anyone to follow the steps, e.g. information on software installed, information on format of input data, missing key steps in the sequence
    \item \textbf{good (80-94):} Complete sequence of commands interleaved with comments, but comments could be improved to provide more guidance
    \item \textbf{excellent (95-100):} Fully explained sequence of commands interleaved with comments which make the whole analysis easy to follow and reproduce; details on software installed and versions installed as well as necessary format for the data input files (or links to input data files if data is publicly available)
\end{itemize}



\subsubsection*{Grading rubric}

\begin{table}[h]
    \begin{tabular}{c|c}
        Grade & Threshold \\
        A & 95 \\
        AB & 80 \\
        B & 70 \\
        C & 50
    \end{tabular}
\end{table}


\subsubsection*{Optional Textbooks}
\begin{itemize}
    \item Phylogenetics in the Genomic Era (open access book: \url{https://hal.inria.fr/PGE/page/table-of-contents}) by Celine Scornavacca, Frederic Delsuc and Nicolas Galtier
    \item Tree thinking: an introduction to phylogenetic biology by David Baum and Stacey Smith
\end{itemize}


\subsubsection*{Class assessments and key dates}
The class will not have any formal assessments other than the final reproducible script (due on May 4, 2026). Late submissions are not allowed.


\subsubsection*{Academic Policies}
\begin{itemize}
    \item The course syllabus provides a general plan for the course; deviations may be necessary
    \item Quarantine or Isolation due to COVID-19: Students should continually monitor themselves for COVID-19 symptoms and get tested for the virus if they have symptoms or have been in close contact with someone with COVID-19. Students should reach out to instructors as soon as possible if they become ill or need to isolate or quarantine, in order to make alternate plans for how to proceed with the course. Students are strongly encouraged to communicate with their instructor concerning their illness and the anticipated extent of their absence from the course (either in-person or remote). The instructor will work with the student to provide alternative ways to complete the course work.
    \item Recordings: Lecture materials and recordings of this course are protected intellectual property at UW-Madison. Students in this course may use the materials and recordings for their personal use related to participation in this class. Students may also take notes solely for their personal use. If a lecture is not already recorded, you are not authorized to record lectures without permission unless you are considered by the university to be a qualified student with a disability requiring accommodation. [Regent Policy Document 4-1] Students may not copy or have lecture materials and recordings outside of class, including posting on internet sites or selling to commercial entities. Students are also prohibited from providing or selling their personal notes to anyone else or being paid for taking notes by any person or commercial firm without the instructor's express written permission. Unauthorized use of these copyrighted lecture materials and recordings constitutes copyright infringement and may be addressed under the university's policies, UWS Chapters 14 and 17, governing student academic and non-academic misconduct.
    \item Evaluations: Your constructive assessment of this course plays an indispensable role in shaping education at UW-Madison. Upon completing the course, please take the time to fill out the online course evaluation.
    \item Academic Integrity: By enrolling in this course, each student assumes the responsibilities of an active participant in UW-Madison's community of scholars in which everyone's academic work and behavior are held to the highest academic integrity standards. Academic misconduct compromises the integrity of the university. Cheating, fabrication, plagiarism, unauthorized collaboration, and helping others commit these acts are examples of academic misconduct, which can result in disciplinary action. This includes but is not limited to failure on the assignment/course, disciplinary probation, or suspension. Substantial or repeated cases of misconduct will be forwarded to the Office of Student Conduct and Community Standards for additional review.
    \item Accommodations for Students with Disabilities: McBurney Disability Resource Center syllabus statement: The University of Wisconsin-Madison supports the right of all enrolled students to a full and equal educational opportunity. The Americans with Disabilities Act (ADA), Wisconsin State Statute (36.12), and UW-Madison policy (Faculty Document 1071) require that students with disabilities be reasonably accommodated in instruction and campus life. Reasonable accommodations for students with disabilities is a shared faculty and student responsibility. Students are expected to inform faculty of their need for instructional accommodations by the end of the third week of the semester, or as soon as possible after a disability has been incurred or recognized. Faculty will work either directly with the student or in coordination with the McBurney Center to identify and provide reasonable instructional accommodations. Disability information, including instructional accommodations as part of a student's educational record, is confidential and protected under FERPA.
    \item Diversity and Inclusion: Institutional Statement on Diversity: Diversity is a source of strength, creativity, and innovation for UW-Madison. We value the contributions of each person and respect the profound ways their identity, culture, background, experience, status, abilities, and opinion enrich the university community. We commit ourselves to the pursuit of excellence in teaching, research, outreach, and diversity as inextricably linked goals. The University of Wisconsin-Madison fulfills its public mission by creating a welcoming and inclusive community for people from every background -- people who as students, faculty, and staff serve Wisconsin and the world.
    \item Religious Observances: UW faculty policy states that mandatory academic requirements should not be scheduled on days when religious observances may cause substantial numbers of students to be absent. Refer to the university’s Academic Calendar for specific information.
\end{itemize}

\end{document}